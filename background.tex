\section{Background}\label{sec:background}
The location the thrown object (for this document it will be a ball) lands is determined by the velocity (magnitude and direction) of the end-effector and the location of the ball at the release point.  There are many examples of throwing/pitching machines made by commercial companies such as Louisville Slugger $^{TM}$, Jugs$^{TM}$, and Atec$^{TM}$ to name a few.  These devices typically contain one to two fly wheels that the ball travels through in order to be launched or a spring loaded arm that is compressed and released.  Robots designed for throwing come in many shapes and sizes depending on the objective.  2-DOF mechanisms are able to throw in $R^3$ space with the correct kinematic structure.  Such a mechanism can choose its release point or its end-effector velocity but not both.  Mechanisms containing 3 or more DOF with the correct kinematic structure are able to throw in $R^3$ and choose both the release point and the end-effector velocity simultaneously.

% need to modify
Low degree of freedom throwing machines/robots are common.  Typical throwing robots have between one and three degrees of freedom (DOF)~\cite{509405, Lynch97dynamicnonprehensile, 5152525, 509335, springerlink:10.1007/s10015-006-0401-0}.  All of these mechanisms are limited to throwing in a plane.   Sentoo et al.~\cite{4651142} achieved an end-effector velocity of 6.0 m/s and can throw in $R^3$ space using it's Barret Technology Inc 4-DOF arm with a $360^o$ rotation base yaw actuator.  These low degree of freedom throwing robots are either physically attached/planted to the mechanical ground or have a base that is significantly more massive then the arm. 

Haddadin et al.\cite{6094757} used their 7-DOF arm and a 6-DOF force torque sensor with standard feedback methods to dribble a basket ball.  In addition Zhikun et al.~\cite{6094892} used reinforcement learning to teach their 7-DOF planted robot arm to play ping-pong.  Likewise Schaal et al.~\cite{schaal01/BIRG} taught their high degree of freedom (30-DOF) humanoid robot to hit a tennis ball using on-line special statistical learning methods.  Visual feedback was used in the basketball throwing robot by Hu et al.~\cite{5649335} achieving accuracy an of 99\%.  All of the latter robots were fixed to the ground to guarantee stability.

Kim et al. \cite{5686315,JooH2011438} takes the research to the next level with finding optimal overhand and sidearm throwing motions for a high degree of freedom humanoid computer model.  The model consists of 55-DOF and is not fixed to mechanical ground or a massive base.  Motor torques are then calculated to create both sidearm and overhand throws that continuously satisfies the zero-moment-point stability criteria~\cite{4309277}.  

The highly articulated 40-DOF adult size humanoid robot Jaemi Hubo KHR-4 (Fig.~\ref{fig:huboOneFoot}) is the platform focused on in this work.  Jaemi Hubo is a high-gain, position-controlled biped humanoid robot weighing 37kg and standing 130cm tall.  It is designed and made by Dr. Jun-Ho Oh director of the Hubo Lab at the Korean Advanced Institute of Science and Technology (KAIST).  Jaemi has been located at the Drexel Autonomous Systems Lab (DASL) at Drexel University since Fall of 2008.  DASL has extensive experience with the Jaemi Hubo KHR-4 platform in key areas needed to complete this work.  Balancing was explored when developing a real-time zero moment point (ZMP) preview control system for stable walking~\cite{5686276}.  A full-scale safe testing environment designed for experiments with Jaemi Hubo was created using DASL's Systems Integrated Sensor Test Rig (SISTR)~\cite{5686325}.  Additionally all algorithms are able to be tested on miniature and virtual versions of Jaemi Hubo prior to testing on the full-size humanoid robot through the creation of a surrogate testing platform for humanoid robots~\cite{5379582}.