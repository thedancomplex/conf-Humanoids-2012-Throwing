\subsection{Throwing Using Sparse Reachable Map}\label{sec:sec:srm}

A Sparse Reachable Map (SRM) is used to create a collision free trajectories while having the end-effector reach a desired velocity\cite{dlofaro-srm}.
The SRM has been shown to be a viable method for trajectory generation for high degree of freedom, high-gain position controlled robots.  This remains true when operating without full knowledge of the reachable area as long as a good collision model of the robot is available. 
The end-effector velocity (magnitude and direction) is specified as well as a duration of this velocity. 
The SRM is created by making a sparse map of reachable end-effector positions in free space and the corresponding poses in joint space is created using random sampling in joint space and forward kinematics. 
The desired trajectory in free space is placed within the sparse map with the first point of the trajectory being a known pose from the original sparse map. 
The Jacobian Transpose Controller method of inverse kinematics is then used to find the subsequent points in the trajectory. 
Each pose in the trajectory is checked against the collision model to guarantee no self-collisions. 