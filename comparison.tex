\section{Method Comparison}\label{sec:comparison}
All three methods described are stable during the throwing motion and can successfully throw a baseball.  Table~\ref{table:comp} shows the end-effector (EEF) velocity, joint failure rate, overhand or underhand throwing and stability of the three different methods of end-effector velocity control used in this paper human-robot joint mapping (MoCap), SRM based and key-frame.

\begin{table}[!t]
%% increase table row spacing, adjust to taste
\renewcommand{\arraystretch}{1.3}
% if using array.sty, it might be a good idea to tweak the value of
% \extrarowheight as needed to properly center the text within the cells
\caption{Comparison between the three different methods of end-effector (EEF) velocity control: human-robot joint mapping (MoCap), SRM based and key-frame}
\label{table:comp}
\centering
\begin{tabular}{|c|c|c|c|c|}
\hline
  				& EEF 																	& Joint 						& Throw						& Stable  			\\
  				& Velocity ($\frac{m}{s}$)							& Failure (\%)			& Overhand (y/n)	& (y/n)					\\
\hline	
MoCap 		& 4.0																		& 0									& n								& y 						\\
\hline
SRM 			& 4.9																		& 10								& y								& y							\\
\hline
Key-Frame & 4.8 																	& 0 								& y								&	y							\\
\hline
\end{tabular}
\end{table}

The SRM technique worked well however the large jerk on each of the joints created large torques caused the motor controller to over torque/current and shutdown 10\% of the time.  
The pitch needs to work 100\% of the time thus this method is not well suited for the event.  
Both the human-robot joint mapping via MoCap and key-frame based methods consistently worked and stayed stable.  A secondary objective is to have the throwing motion be overhand like a standard Major League Baseball pitcher.  
Due to the constraints placed on the joint mapping of the MoCap method in Section~\ref{sec:sec:mocap} an over arm throw would be impossible to preform due to the robot's $\pm180^o$ joint limitations.  The key-frame method was chosen as the method to throw the ball.  Section~\ref{sec:finalDesign} describes the modifications to the system to allow it to reliably throw the pitch the desired distance.